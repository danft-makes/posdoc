\documentclass{article}

% Language setting
% Replace `english' with e.g. `spanish' to change the document language
\usepackage[english]{babel}

% Set page size and margins
% Replace `letterpaper' with `a4paper' for UK/EU standard size
\usepackage[letterpaper,top=2cm,bottom=2cm,left=3cm,right=3cm,marginparwidth=1.75cm]{geometry}

% Useful packages
\usepackage{amsmath}
\usepackage{amsthm}
\usepackage{amssymb}
\usepackage{graphicx}
\usepackage{tikz-cd}
\usepackage[colorlinks=true, allcolors=blue]{hyperref}

\newtheorem{theorem}{Theorem}[section]
\newtheorem{example}{Example}[section]
\newtheorem{corollary}{Corollary}[theorem]
\newtheorem{lemma}[theorem]{Lemma}
\newtheorem{proposition}[theorem]{Proposition}


\title{Your Paper}
\author{You}

\begin{document}
\maketitle

\begin{abstract}
Your abstract.
\end{abstract}

\section{Introduction}
\subsection{Pre-req}
Let $f$ be a homogeneous polynomial in $\mathbb{C}[x,y,z]$ of degree $d+1$ defining a curve $C = V(f)$ and $\delta = P\partial_x+Q\partial_y+R\partial_z$ be a derivation with $P,Q,R$ of the same degree $a$. Assume that $\delta(f) \in Der(f)=\{\delta | \delta(f) \in (f)\}$, without loss of generality we can take $\delta(f)=0$, that is $\delta \in Der_0(f)=\{\delta | \delta(f)=0\}$. Let $\mathcal{T}_f$ be the sheaf of logarithmic derivations of $f$ defined by the exact sequence
\begin{equation}
    0 \to \mathcal{T}_f \to \mathcal{O}_{\mathbb{P}^2}^{\oplus 3} \to \mathcal{I}_J(d) \to 0
\end{equation}
where $\mathcal{I}_J$ is the ideal sheaf defined by the three partial derivatives of $f$.
If $\delta$ is a minimal degree derivation then we have an induced section $\tilde{\delta} \in H^0(\mathcal{T}_f(a))$ and a scheme of points $Z$ satisfying the following diagram:

\begin{center}
	\begin{tikzcd}[ampersand replacement=\&]

		\mathcal{O}_{\mathbb{P}^2}(-a) \ar[equal]{r} \ar[hookrightarrow,"\tilde{\delta}"]{d} \& \mathcal{O}_{\mathbb{P}^2}(-a) \ar[hookrightarrow,"\delta"]{d}  \\
		\mathcal{T}_f \arrow[hookrightarrow]{r} \ar[twoheadrightarrow]{d} \& \mathcal{O}_{\mathbb{P}^2}^{\oplus 3} \ar[rr,bend left, shift right,"\nabla f"] \ar[twoheadrightarrow,"\alpha"]{r} \arrow[twoheadrightarrow]{d} \& \mathcal{I}_{J}(d) \ar[hookrightarrow]{r} \ar[equal]{d} \& \mathcal{O}_{\mathbb{P}^2}(d) \\
		\mathcal{I}_Z(a-d) \arrow[hookrightarrow]{r} \& F \arrow[twoheadrightarrow]{r}\& \mathcal{I}_{J}(d)
	\end{tikzcd}
\end{center}
The length of $Z$ can be computed from the second Chern class $c_2(\mathcal{T}_f(a))$ with the additional knowledge that $c_2(\mathcal{T}_f) = d^2 - \sum_{p \in Sing(C)} \tau_p(C)$ where $\tau_p(C)$ is the Tjurina number of the singularity $p$ (the sum for every singular point is called the total Tjurina number).

\begin{lemma}\label{smooth maximal bourbaki}
    If $C$ is smooth then $a = d$ and $n = d^2$
\end{lemma}
\begin{proof}
    By dualizing the sequence $$0 \to \mathcal{T}_f \to \mathcal{O}_{\mathbb{P}^2}^{\oplus 3} \to \mathcal{O}_{\mathbb{P}^2}(d) \to 0$$
    we obtain
    \begin{equation}
        0 \to \mathcal{O}_{\mathbb{P}^2}(-d) \to \mathcal{O}_{\mathbb{P}^2}^{\oplus 3} \to \mathcal{T}_f(d) \to 0
    \end{equation}
    which implies that $H^0(\mathcal{T}_f(d-1))=0$ and $H^0(\mathcal{T}_f(d))\not=0$ so $a = d$ and $n = d^2$.
\end{proof}
\begin{lemma}
Let $\mathcal{O}_{Z(\delta)}$ be the structure sheaf of the zeroes of the three homogeneous forms defining $\delta$ (we will denote this scheme as $Z(\delta)$). Then:
    \begin{equation}
        \mathcal{E}xt^1(F,\mathcal{O}_{\mathbb{P}^2}) \cong \mathcal{O}_{Z(\delta)}
    \end{equation}
\end{lemma}
\begin{proof}
    Dualizing the second column gives us the exact sequence:
    \begin{equation}
        0 \to F^* \to \mathcal{O}_{\mathbb{P}^2}^{\oplus 3} \to \mathcal{O}_{\mathbb{P}^2}(a) \to \mathcal{E}xt^1(F,\mathcal{O}_{\mathbb{P}^2}) \to 0
    \end{equation}
    which implies that $\mathcal{E}xt^1(F,\mathcal{O}_{\mathbb{P}^2})$ is a quotient of $\mathcal{O}_{\mathbb{P}^2}(a)$ by the image $\Phi$ of $\mathcal{O}_{\mathbb{P}^2}^{\oplus 3} \to \mathcal{O}_{\mathbb{P}^2}(a)$. Analyzing the support of the exact sequence $$0 \to F^* \to \mathcal{O}_{\mathbb{P}^2}^{\oplus 3} \to \Phi \to 0$$ shows us that $\Phi$ is zero exactly at $Z(\delta)$, hence it corresponds to $\mathcal{I}_{Z(\delta)}(a)$.
\end{proof}

\begin{lemma}
    $Z$ is contained in $Z(\delta)$.
\end{lemma}
\begin{proof}
By dualizing the third row we obtain
\begin{equation}
    0 \to \mathcal{O}_{\mathbb{P}^2}(-d) \to F^* \to \mathcal{O}_{\mathbb{P}^2}(d-a) \to \mathcal{E}xt^1(\mathcal{I}_J(d),\mathcal{O}_{\mathbb{P}^2}) \to \mathcal{E}xt^1(F,\mathcal{O}_{\mathbb{P}^2}) \to \mathcal{E}xt^1(\mathcal{I}_Z(a-d),\mathcal{O}_{\mathbb{P}^2}) \to 0
\end{equation}
which simplifies into
\begin{equation}
    0 \to \mathcal{O}_{\mathbb{P}^2}(-d) \to F^* \to \mathcal{O}_{\mathbb{P}^2}(d-a) \to \omega_J \to \mathcal{O}_{Z(\delta)} \to \omega_Z \to 0
\end{equation}
and hence the support of $\omega_Z$ is contained in $Z(\delta)$.
\end{proof}

\subsection{Addition of a line}
Now let $l = l_1*x+l_2*y+l_3*z$ be the equation of a line $L$ with $l_i \in \mathbb{C}$. We can construct a derivation of degree $a+1$ in $Der_0(f*l)$ from $Der_0(f)$ resulting in a morphism:
\begin{equation}
    \delta \mapsto l\delta := l*\delta - \frac{\delta(l)*\delta_E}{d+2}
\end{equation}
where $\delta_E = x*\partial_x + y*\partial_y + z*\partial_z$ is the Euler derivation. There is then the following commutative diagram:
\begin{center}
\begin{tikzcd}[ampersand replacement=\&]
		\mathcal{O}_{\mathbb{P}^2}(-a-1) \ar[equal]{r} \ar[hookrightarrow,"\tilde{\delta}"]{d} \& \mathcal{O}_{\mathbb{P}^2}(-a-1) \ar[hookrightarrow,"l\delta"]{d}  \\
		\mathcal{T}_f(-1) \arrow[hookrightarrow]{r} \ar[twoheadrightarrow]{d} \& \mathcal{T}_{f*l} \ar[twoheadrightarrow]{r} \arrow[twoheadrightarrow]{d} \& \mathcal{O}_{L}(a-d-|Z'\cap L|) \ar[equal]{d} \\
		\mathcal{I}_Z(a-d-1) \arrow[hookrightarrow]{r} \& \mathcal{I}_{Z'}(a-d) \arrow[twoheadrightarrow]{r}\& \mathcal{O}_{L}(a-d-|Z'\cap L|)
\end{tikzcd}
\end{center}
which implies that $Z'$ can be divided by points in the line $L$ and points outside the line, given by $Z$. An interesting fact is that the points in $Z' \cap L$ must be inside $Z(\delta(l))$, while $Z = Z' \setminus L$ are necessarily in the eigenscheme of $\delta$:
\begin{lemma}
    $Z' \setminus Z$ is contained in $Z(\delta(l))$ and $Z$ is in the eigenscheme of $\delta$. Furthermore, $Z(\delta(l)) \subset Z(\delta)$.
\end{lemma}
\begin{proof}
We start by remembering that any point in $Z'$ must be in the vanishing of the three homogeneous polynomials constituting $l\delta$, so let $p = (p_1;p_2;p_3) \in Z' \subset V(l\delta)$. We have that $0 = (l*\delta - \frac{\delta(l)*\delta_E}{d+2})|_p$, suppose first that $p$ is not in $Z$ and thus $p \in L$ so that $l(p)=0$. Then $0 = \frac{\delta(l)*\delta_E}{d+2}|_p = \frac{\delta(l)(p)}{d+2}(p_1,p_2,p_3)$ which implies that $\delta(l)(p) = 0$ so $p$ is in $V(\delta(l))$. Now if $p \in Z$ then $p \in Z' \setminus L$, hence $l(p) = \alpha \not = 0$ and $\alpha*(P(p),Q(p),R(p)) = \frac{\delta(l)(p)}{d+1}*(p_1,p_2,p_3)$. We have two cases to consider: If $\delta(l)(p) = 0$ then $P(p) = Q(p) = R(p) = 0$ and if $\delta(l)(p) \not = 0$ we would have that $p$ is a fixed point of the rational map induced by the derivation $\delta$. Either case implies that $p$ is in the eigenscheme of $\delta$.

To prove that $Z(\delta(l)) \subset Z(\delta)$ we consider the following diagram:
\begin{center}
\begin{tikzcd}[ampersand replacement=\&]
		\& \mathcal{O}_{\mathbb{P}^2}(-a) \ar[equal]{r} \ar[hookrightarrow,"\delta"]{d} \& \mathcal{O}_{\mathbb{P}^2}(-a) \ar[hookrightarrow,"\delta(l)"]{d}  \\
		\mathcal{T}_f \arrow[hookrightarrow]{r} \ar[equal]{d} \& \mathcal{O}_{\mathbb{P}^2}^{\oplus 3} \ar[twoheadrightarrow,"\nabla(l)"]{r} \arrow[twoheadrightarrow]{d} \& \mathcal{O}_{\mathbb{P}^2} \ar[twoheadrightarrow]{d} \\
		\mathcal{T}_f \arrow[hookrightarrow]{r} \& F \arrow[twoheadrightarrow]{r}\& \mathcal{O}_{Z(\delta(l))}
\end{tikzcd}
\end{center}
By comparing the supports in the third row we conclude that $Z(\delta(l))$ is in $supp(F)$, which is $Z(\delta)$.
\end{proof}

\begin{corollary}
    If $\mathcal{T}_f$ is free then $Z' \subset Z(\delta(l)) \cap L$
\end{corollary}
\begin{proof}
    Since $\mathcal{T}_f$ is free we have that $Z$ is empty and $Z' \subset L$. Any point $p \in Z'$ must be in the zero set of the section $l\delta = l*\delta - \frac{\delta(l)}{d+2}\delta_E$, but since $l(p) = 0$ we have that $\frac{\delta(l)(p)}{d+2}\delta_E(p) = \frac{\delta(l)(p)}{d+2}(p_1,p_2,p_3) = 0$ so $\delta(l)(p) = 0$.
\end{proof}

\begin{corollary}
    Let $L$ be a generic line so that it intersects the curve $X$ given by $f$ only in simple points. Then:
    \begin{itemize}
        \item $\mathcal{T}_f$ free implies that $Z'$ has length $a$
        \item $X$ smooth implies that $Z'$ has length $h^0(\mathcal{O}_Z) + d = d^2+d$
    \end{itemize}
\end{corollary}
\begin{proof}
    The computation of $c_2 \mathcal{T}_f(a)$ and $c_2 \mathcal{T}_{f*l}(a+1)$ can be simplified in the following way: Let $n = h^0(\mathcal{O}_Z)$, $n' = h^0(\mathcal{O}_Z)$, $j = h^0(\mathcal{O}_J)$ and $j' = h^0(\mathcal{O}_{J'})$. Then $n+j = d^2 - a*d + d^2$ and $n'+j' = (d+1)^2 - (a+1)*(d+1) + (a+1)^2$.

    Since $L$ contributes with $d+1$ simple points in any case, we know how the total Tjurina number behaves: we have that $j' = j + d + 1$. So by subtracting one equation from the other we have that $n'-n +d+1= -d^2+a*d-a^2 + d^2+2d+1 - ad - a - d - 1 + a^2+2a+1 = -a^2+2d+1-a-d-1+a^2+2a+1 = d+a+1$ so that $n'-n = a$. The first case is when $n=0$ so that $n' = a$, the second case happens when $n = d^2$ and $a = d$ so that $n' = d^5+d$.
\end{proof}

\textcolor{red}{Possible question: Are all $a$ points the entire $L \cap Z(\delta(l))$????}

\begin{proposition}
    If $\mathcal{T}_f$ \textcolor{blue}{or better, the additional points are given by:} is free then $Z' = L \cap Z(\delta(l))$
\end{proposition}
\begin{proof}
    \textcolor{red}{TODO!!!!!}
\end{proof}

\begin{example}
    Consider the union of two conics sharing a unique tangent given by $x = 0$, we can write explicitly $f = (z^2-x*y)*(z^2-x*y+x*z)*(z^2-x*y-x*z)$. Denote $X = V(f)$ and let $L = V(a*x+b*y+c*z)$ be a general line.

    A minimal derivation of $f$ has degree $2$ and can be found by computing the canonical derivation of the pencil of two conics sharing one tangent, say $\delta = P*\partial_x + Q*\partial_y + R*\partial_z$ with $P,Q,R$ of degree $2$. Explicitly we have that $P = -x^2, Q = 2*z^2+x*y, R = x*z$.

    We can also compute the length of $Z$ and $Z'$ with the additional knowledge of the Tjurina numbers for the singularities of $X \cup L$ and conclude that $X$ is free (hence $Z$ is empty) and $Z'$ has length 2. So we have the following diagram:
    \begin{center}
    \begin{tikzcd}[ampersand replacement=\&]
    		\mathcal{O}_{\mathbb{P}^2}(-3) \ar[equal]{r} \ar[hookrightarrow,"\tilde{\delta}"]{d} \& \mathcal{O}_{\mathbb{P}^2}(-3) \ar[hookrightarrow,"l\delta"]{d}  \\
    		\mathcal{T}_f(-1) \arrow[hookrightarrow]{r} \ar[twoheadrightarrow]{d} \& \mathcal{T}_{f*l} \ar[twoheadrightarrow]{r} \arrow[twoheadrightarrow]{d} \& \mathcal{O}_{L}(-3-|Z'\cap L|) \ar[equal]{d} \\
    		\mathcal{O}_{\mathbb{P}^2}(-4) \arrow[hookrightarrow]{r} \& \mathcal{I}_{Z'}(-3) \arrow[twoheadrightarrow]{r}\& \mathcal{O}_{L}(-3-|Z'\cap L|)
    \end{tikzcd}
    \end{center}
The third row implies that $Z'$ is contained in $L$ so that $|Z' \cap L| = 2$. By the lemma above we also conclude that $Z' \subset V(\delta(l)) = V(a*P+b*Q+c*R)$, so it is possible to provide a form for both points in $Z'$ in terms of $a,b,c$ by considering the intersection of $V(\delta(l))$ and $L$:

\textcolor{red}{Im working in the affine patch $z=1$ because in my tests with code, $Z'$ never touched the line $z=0$. But I'm not sure how to argue that here.}

We work in the affine patch $z=1$, $a*P + b*Q + c*R = 0$ implies that $-a*x^2+x*(c+b*y)+2*b = 0$. Using the equation of the line $b*y=-a*x-c$ we find $-2*a*x^2+2*b=0$ and hence (for a general line) $Z' = \{(\sqrt{\frac{b}{a}}:\frac{-\sqrt{a*b}-c}{b}:1),(-\sqrt{\frac{b}{a}}:\frac{\sqrt{a*b}-c}{b}:1)\}$. Denote the first point by $C$ and the second point by $D$.

There are two distinguished lines for this example: $x = 0$ and $z = 0$. The first is the unique tangent for the singular point $(0:1:0)$ and the second is the unique line passing through both singularities $(0:1:0),(1:0:0)$. The intersection of each of those two lines with $L$ gives then two points $A = (0:-c:b)$ and $B = (-b:a:0)$ in $L$. By projecting $A,B,C,D$ to the line $z=0$ we can compute the cross-ratio $CR(A,B;C,D) = \frac{AC}{BC}*\frac{BD}{AD}$:
\begin{equation}
    CR(A,B;C,D) = \frac{\sqrt{\frac{b}{a}}}{\sqrt{a*b}+c-\sqrt{a}*\sqrt{b}}*\frac{-\sqrt{a*b}+c+\sqrt{a*b}}{-\sqrt{\frac{b}{a}}} = -1
\end{equation}
So the cross-ratio is constant for any $a,b,c$, provided that $a \not = 0$ and $b \not = 0$. (\textcolor{red}{I think there is a better argument here instead of requiring the non-vanishing conditions of a,b...}).
\end{example}

\begin{example}
    Consider a pencil of two conics sharing two tangents given by $C_1 = x^2-z^2$ and $C_2 = y^2$ and pick the union of two smooth conics belonging to the arrangement. Let $L = V(a*x+b*y+c*z)$ be a general line. Then the canonical derivation is given by $\delta' = y*\delta$ and hence the minimal derivation is $\delta = P*\partial_x + Q*\partial_y + R*\partial_z$ with $P,Q,R$ of degree $1$ which implies that $\delta(l) = a*P + b*Q + c*R$ is also of degree $1$. Using the fact that the Tjurina number for each singularity is $3$ we conclude that the length of $Z$ is $c_2(\mathcal{T}_C(1)) = 1$, the same can be done for $Z'$ since we just need to add each intersection with $L$ in the total Tjurina number. If $L$ is generic we conclude that $Z'$ has length $c_2(\mathcal{T}_{C \cup L}(2)) = 2$. To understand the geometric configuration of $Z$ and $Z'$ we consider the following commutative diagram:
    \begin{center}
    \begin{tikzcd}[ampersand replacement=\&]
    		\mathcal{O}_{\mathbb{P}^2}(-2) \ar[equal]{r} \ar[hookrightarrow,"\tilde{\delta}"]{d} \& \mathcal{O}_{\mathbb{P}^2}(-2) \ar[hookrightarrow,"l\delta"]{d}  \\
    		\mathcal{T}_f(-1) \arrow[hookrightarrow]{r} \ar[twoheadrightarrow]{d} \& \mathcal{T}_{f*l} \ar[twoheadrightarrow]{r} \arrow[twoheadrightarrow]{d} \& \mathcal{O}_{L}(-2-|Z'\cap L|) \ar[equal]{d} \\
    		\mathcal{I}_Z(-3) \arrow[hookrightarrow]{r} \& \mathcal{I}_{Z'}(-2) \arrow[twoheadrightarrow]{r}\& \mathcal{O}_{L}(-2-|Z'\cap L|)
    \end{tikzcd}
    \end{center}
    Using the third row of the diagram and the previous lemma we can also conclude that the point $Z$ lies outside of $L$ and hence $Z \cap L = 1$ for a generic line $L$. Let $L_{\delta}$ be the line defined by $\delta(l)$, the intersection $L_{\delta} \cap L$ is precisely the unique point in $Z' \setminus Z$. We can compute this point using the explicit form for the minimal derivation:

    $$P = z, Q = 0, R = x \implies \delta(l) = a*z+c*x$$
    $$a*x+b*y+c*z = 0 \implies L \cap L_{\delta} = \{(\frac{-a}{c}:\frac{-(c^2-a^2)}{bc}:1) \}$$

    \textcolor{red}{Now the splitting, I'm not sure where to put it}
    Now, $|Z' \cap L| = 1$ implies that there is a surjection:
    \begin{equation}
        {\mathcal{T}_{f*l}}|_L \to \mathcal{O}_L(-3) \to 0
    \end{equation}
    Since $c_1(\mathcal{T}_{f*l}) = -4$ we need only to consider the possible splittings $(-1,-3)$ and $(0,-4)$. If we had ${\mathcal{T}_{f*l}}|_L \cong \mathcal{O}_L \oplus \mathcal{O}_L(-4) \to \mathcal{O}_L(-3)$ there would be a cokernel corresponding to a point in $L$. But since this morphism is a surjection we conclude this case does not occur, so we have ${\mathcal{T}_{f*l}}|_L \cong \mathcal{O}_L(-1) \oplus \mathcal{O}_L(-3)$.
\end{example}
\begin{example}
    Consider our last example with $L$ a sufficiently generic line passing through the unique point $Z$ we still have that the length of $Z'$ is $2$.
    \textcolor{red}{I'm a bit confused here because of the third exact row}
    \begin{equation}
        0 \to \mathcal{I}_Z(-3) \to \mathcal{I}_{Z'}(-2) \to \mathcal{O}_L(-2-|Z'\cap L|) \to 0
    \end{equation}
    \textcolor{red}{Since Z is in L by construction the exact sequence above seems to be in contradiction. Because it is saying that L passes through $Z'$ but does not intersect with $Z$. I think the contradiction is I assumed that $l\delta$ is a minimal derivation so we conclude that $\delta$ itself is a minimal derivation for $C \cup L$ and thus it is free?}

    \textcolor{red}{The original purpose of the example 1.3 was to look at $L_{\delta} = \delta(l)$ to conclude that $\mathcal{T}_{f*l}|_{L_{\delta}} = \mathcal{O}_{L_{\delta}}(-4) \oplus \mathcal{O}_{L_{\delta}}$ as we spoke in our last meeting.}
\end{example}

\begin{proposition}
    Let $X = V(f)$ be a free curve with $\mathcal{T}_f \cong \mathcal{O}_{\mathbb{P}^2}(-a) \oplus \mathcal{O}_{\mathbb{P}^2}(-b)$ where $a \leq b$ and suppose that $l$ is a linear form such that $H^0(\mathcal{T}_{l*f}(a)) \not = 0$. Then $\mathcal{T}_{l*f}$ is free.
\end{proposition}
\begin{proof}
    Let $\xi$ be a derivation of degree $a$ such that $\xi(l*f) = 0$. The Leibniz rule implies that $f*\xi(l) = -l*\xi(f)$ so $\xi(f) \in (f)$ and hence $\xi$ is in $Der(f)$ (\textcolor{red}{and $\xi(f) = -\frac{\xi(l)}{l}*f$}). There is a canonical morphism given by $\mathcal{T}_{l*f} \to \mathcal{T}_{f}$ sending $\xi$ to $\xi - \frac{\xi(f)}{f*deg(f)}*\delta_E$
\end{proof}

\end{document}
